\documentclass[12pt]{article}
\usepackage{geometry} % to change the page dimensions
\usepackage{graphicx} % to include images
\usepackage{hyperref} % for hyperlinks
\usepackage[backend=biber,style=apa]{biblatex}

% Adjust the page margins
\geometry{a4paper}

\title{
    \textbf{Final Project Proposal: \\Short-term water demand prediction} \\
    \vspace{0.5em}
    \large Project Category: General Machine Learning
}

\author{
    Haocheng Fan, hfan11 \\
}

\date{\today}

\begin{document}
\maketitle

\section*{Motivation}
My homwtown Shaoxing, often labeled as the "Eastern Venice" of China, is witnessing profound challenges in water resource management. 
As climate unpredictability intensifies, the reliability of available water resources is becoming increasingly vital. 
This urgency is further exacerbated by technological progression and population surge, escalating water demand complexities. 
Addressing these multifaceted challenges, especially in the realm of optimizing operational costs and energy efficiency amidst water supply uncertainties, 
is imperative. Above all, we aim to explore the potential of machine learning in devising short-term water demand forecasting mechanisms to ensure judicious water utilization scheduling.

\section*{Method}

We intend to evaluate three distinct machine learning methods to identify the best-performing model for our specific context. These methods include the Linear Regression Model (LRM), Decision Tree Model (DTM), and Support Vector Regression Model (SVRM).
 

\section*{Intended Experiments}
\begin{enumerate}

    \item \textbf{Data Collection}


    \item \textbf{Data Pre-processing:}
    \begin{itemize}
        \item Remove the outliers from the data.
        \item Select reasonable variables for the models.
        \item Construct the feature matrix.
    \end{itemize}

    \item \textbf{Correlation Analysis}
    
    \item \textbf{Model Preparation:}
    \begin{itemize}
        \item Set up the three machine learning model algorithms.
    \end{itemize}

    \item \textbf{Model Training:}
    \begin{itemize}
        \item Train the models using the feature matrix.
    \end{itemize}

    \item \textbf{Model Evaluation:}
    \begin{itemize}
        \item Evaluate the models using the following metrics:
        \begin{itemize}
            \item Mean Squared Error ($MSE$)
            \item Root Mean Squared Error ($RMSE$)
            \item Mean Absolute Error ($MAE$)
            \item Coefficient of Determination ($R^2$)
        \end{itemize}
    \end{itemize}

    \item \textbf{Choose the best model for water demand forecast}
\end{enumerate}


\section*{Optional: Dataset and Prior Research}
\textbf{Relevant Dataset:} Provide a pointer to a dataset that might be relevant to your project. \\
\textbf{Prior Research:} \\
Q, Shuang \& Zhao, RT. (2021). Water Demand Prediction Using Machine Learning Methods: A Case Study of the Beijing–Tianjin–Hebei Region in China. \textit{Water}, 13(3), 310. https://doi.org/10.3390/w13030310.\\


Kavya, M., Mathew, Aneesh, Shekar, Padala Raja & Sarwesh, P. (2023). Short term water demand forecast modelling using artificial intelligence for smart water management. \textit{Sustainable Cities and Society}, 95, 104610. https://doi.org/10.1016/j.scs.2023.104610.



\end{document}
